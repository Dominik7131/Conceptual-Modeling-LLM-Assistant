\chapter{LLM-based modeling assistant framework}
\label{chap:framework}

In this chapter, we define a formal LLM-based modeling assistant framework consisting of different types of generators that can be implemented with different prompting techniques and domain description pre-processing techniques. Furthermore, we define the requirements and the basic meta-structure for these generators.


\section{Approach}

We propose solving the automation of the domain modeling steps from the section \ref{sec:modeling_steps} as text generation problems similar to \citet{Chen2023}, where the authors consider a domain description $T$ with an underlying ground truth model $M$, and approach the problem of model generation as the problem of defining a text generator $f$ with $M' = f(T)$, where $M'$ is the generated model similar to $M$. The generator $f$ is an LLM-based operator defined by a predesigned prompt that instructs an LLM to return a structured textual representation of $M'$ for the input $T$. Based on the generator $f$, we define more fine-grained text generation problems solved by the \emph{text to model generators} that can be used in combination with the \emph{auxiliary generators}.


\section{Text to model generators}

These generators solely based on the given domain description suggest some part of the domain model. We divide these generators into the following categories.


\subsection{Domain element generators}

The domain element generators are used to generate suggestions for classes, attributes, and associations solely based on the given domain description. Subsequently, these suggestions can be used to \emph{design a class}, \emph{design an attribute for a class}, and \emph{design an association for a class} as described in the domain modeling steps section \ref{sec:modeling_steps}.

\begin{description}
\item [Class generator] $gen_c$ that for $T$ suggests classes $\{C_1, \ldots, C_n\}$ each with a suggested $name(C_i)$.

\item [Attribute generator] $gen_a$ that for $T$ and $name(C)$ of a class $C$ suggests attributes $\{P_1, \ldots, P_n\}$, each with $source(P_i) = C$, suggested $name(P_i)$, and original text $orig(P_i)$ that is part of $T$ on which base $gen_a$ suggested $P_i$.

\item [Association generator 1] $gen_{r1}$ that for $T$ and $name(C)$ of a class $C$ suggests associations $\{P_1, \ldots, P_n\}$, each with $source(P_i) = C$ (or $target(P_i) = C$), suggested $name(P_i)$, and with the original text $orig(P_i)$ that is part of $T$ on which base $gen_{r1}$ suggested $P_i$. It also suggests the other class $D$, with $target(P_i)= D$ (or $source(P_i) = D$), and suggested $name(D)$.

\item [Association generator 2] $gen_{r2}$ that for $T$, $name(C)$ of a source class $C$, and $name(D)$ of a target class $D$ suggests associations $\{P_1, \ldots, P_n\}$, each with $source(P_i) = C$, and $target(P_i) = D$, suggested $name(P_i)$, and with the original text $orig(P_i)$ that is part of $T$ on which base $gen_{r2}$ suggested $P_i$.
\end{description}


\subsection{Description generators}

The description generators aim to suggest description for a given class, attribute, and association solely based on the given domain description. Subsequently, these suggestions can be used to \emph{design a description for a class}, \emph{design a description for an attribute}, and \emph{design a description for an association} as described in the domain modeling steps section \ref{sec:modeling_steps}.

\begin{description}
\item [Class description generator] $gen_{cd}$ that for given $T$ and the $name(C)$ of a class $C$ suggests $description(C)$.

\item [Attribute description generator] $gen_{ad}$ that for given $T$, the $name(P)$, and the $name(source(P))$ of an attribute $P$ suggests $description(P)$.

\item [Association description generator] $gen_{rd}$ that for given $T$, the $name(P)$, the $name(source(P))$, and the $name(target(P))$ of an association $P$ suggests $description(P)$.
\end{description}


\subsection{Data type generators}

The data type generator is used to suggest the data type for a given attribute solely based on the given domain description. Subsequently, this suggestion can be used to \emph{design a data type for an attribute} as described in the domain modeling steps section \ref{sec:modeling_steps}.

\begin{description}
\item [Attribute data type generator] $gen_{at}$ that for given $T$, the $name(P)$, and the $name(source(P))$ of an attribute $P$ suggests $dataType(P)$.
\end{description}


\subsection{Cardinality generators}

The cardinality generators aim to suggest cardinalities for a given attribute and association solely based on the given domain description. Subsequently, these suggestions can be used to \emph{design a cardinality for an attribute} and \emph{design a cardinality for an association} as described in the domain modeling steps section \ref{sec:modeling_steps}.

\begin{description}
\item [Attribute cardinality generator] $gen_{ac}$ that for given $T$, the $name(P)$, and the $name(source(P))$ of an attribute $P$ suggests $cardinality(P)$.

\item [Association cardinality generator] $gen_{rc}$ that for given $T$, the $name(P)$, the $name(source(P))$, and the $name(target(P))$ of an association $P$ suggests $cardinality(P)$.
\end{description}


\section{Auxiliary generators}

These generators help with the domain modeling process, for example, by helping to design domain elements and by helping to decide about which next modeling step to use. We divide these generators into the following categories.


\subsection{Original text generators}

When designing a class, an attribute, or an association, the original text generator can be used to suggest a new original text solely based on the given domain description. The original text contains exact parts of the domain description based on which the corresponding domain element can be suggested by the corresponding domain element generator. For example, for the simple domain description \textit{In this company, every employee works in some department} and for the association \textit{works in} with the source class \textit{employee} and the target class \textit{department}, the original text can be \textit{employee works in some department} as this text captures the association in the domain description.

\begin{description}
\item [Class original text generator] $gen_{co}$ that for given $T$ and the $name(C)$ of a class $C$ suggests $orig(C)$.

\item [Attribute original text generator] $gen_{ao}$ that for given $T$, the $name(P)$, and the $name(source(P))$ of an attribute $P$ suggests $orig(P)$.

\item [Association original text generator] $gen_{ro}$ that for given $T$, the $name(P)$, the $name(source(P))$, and the $name(target(P))$ of an association $P$ suggests $orig(P)$.
\end{description}


\subsection{Name generators}

When designing a class, an attribute, or an association, the name generator can be used to suggest a new name for the corresponding domain element.

\begin{description}
\item [Class name generator] $gen_{cn}$ that for given $T$, the $description(C)$, and the $orig(C)$ of a class $C$ suggests  $name(C)$.

\item [Attribute name generator] $gen_{an}$ that for given $T$, the $description(P)$, the $orig(P)$, and the $name(source(P))$ of an attribute $P$ suggests $name(P)$.

\item [Association name generator] $gen_{rn}$ that for given $T$, the $description(P)$, the $orig(P)$, the $name(source(P))$, and the $name(target(P))$ of an association $P$ suggests $name(P)$.
\end{description}


\subsection{Summary generators}

The summary generators are used to generate a description of the selected part of the domain model. These generators can be used, for example, to better understand the semantics of the given domain model, which can help to decide which next modeling step is needed to model the given domain description.

\begin{description}
\item [Summary plain text generator] $gen_{sp}$ that for given style format, names, descriptions, and original texts of classes, attributes, and associations generates an unstructured domain description in the given style format.

\item [Summary descriptions generator] $gen_{sd}$ that for given names, descriptions, and original texts of classes, attributes, and associations generates a domain description in form of a list that contains an item for each given class, attribute, and association.
\end{description}


\section{Prompt templates requirements}

A generator can be executed by an LLM call which requires to define a prompt for each generator. For example, the \emph{class generator} can instruct the LLM with a prompt to generate classes. For inserting the user's input into the prompt, it is required to define a prompt template with placeholders that are later on replaced with the user's input. For example, for the \emph{class generator}, the placeholder for the domain description needs to be introduced to insert the user's domain description.

As the LLM output quality greatly depends on the prompt content and the content's order, it is required to create a well-structured prompt and consider suitable prompting techniques. Additionally, the template should instruct the LLM to work within the given context, provide the required task-specific suggestions, and output them in a predefined format for easy parsing.


\section{Meta-template structure}

Based on these generic requirements, we propose the prompt meta-template shown in figure \ref{fig:meta-templates}.

\begin{figure}[!h]
    \centering
\begin{lstlisting}[
  basicstyle=\ttfamily\scriptsize,
  breaklines=true,
  keywordstyle=\ttfamily,
  stringstyle=\ttfamily,
  showstringspaces=false,
  upquote=true,
  stringstyle=
]
01 {main control instruction}
02 {modeling procedure}
03 {output specification}
04 EXAMPLE START
05 {example main control instruction}
06 {example context specification}
07 Output: {example output}
08 EXAMPLE END
09 {main control instruction}
10 {context specification}
11 Output:
\end{lstlisting}
    \caption{\centering Prompt meta-template for the generators}
    \label{fig:meta-templates}
\end{figure}


A generator template is constructed by replacing the placeholders in the meta-template with generator-specific instructions. The meta-template has the following structure.


%\subsection{Input data name}

%The \emph{input data name} (line 01) provides the names to refer to the data that will be inputted into the prompt, such as the \textit{conceptual model} for generating summaries based on the given conceptual model to let the LLM know solely based on which data it should work with. When examples are provided, the example input is referenced by the same name, and then it is referenced again before the example input data are provided (lines 05 and 06) and the same is done at the end of the prompt (lines 09 and 10) to make the prompt consistent.


\subsection{Main control instruction}

The \emph{main control instruction} (line 01) summarizes the given task, can provide placeholders for some input data, and can reference some input data that are later on provided at the \emph{context specification} (line 10). For example, for the \emph{attribute generator}, the \emph{main control instruction} can state to solely based on the given domain description and the given source class to generate attributes. The source class can be provided in the form of a placeholder by the \emph{main control instruction} and the domain description can be provided in the form of a placeholder in the \emph{context specification}. If examples are provided (lines 04 -- 08), the \emph{main control instruction} can be repeated at the end of the prompt (line 09) to emphasize its importance.


\subsection{Modeling procedure}

The \textit{modeling procedure} (line 02) instructs the LLM on how to proceed before generating the final suggestions. Here, various prompting techniques, such as chain of thoughts, can be implemented.


\subsection{Output specification}

The \textit{output specification} (line 03) defines the required output format so that the output can be automatically parsed.


\subsection{Example specification}

Lines 04 -- 08 represent optional examples. They are all separated from the other instructions with the \texttt{EXAMPLE START} and the \texttt{EXAMPLE END} delimiters. Here, an N-shot prompt design strategy can be implemented, where the prompts contain one or more examples. Each example starts with the \emph{example main control instruction} (line 05), which is the same as the \emph{main control instruction} but with a concrete class $C$ (and in some cases also with a class $D$) if the task is class-specific. The \emph{example context specification} (line 06) is the concrete context specification for the example and the \emph{example output} (line 07) is the expected output for the given context. The \emph{example output} must correspond to the structure of the \emph{output specification} to make the prompt consistent.


\subsection{Context specification}

The context is specified at the end of the meta-template in the \emph{context specification} (line 10). It first states the input data name which for consistency must be the same as the input data name stated by the \emph{main control instruction}. Then it contains a placeholder for the corresponding input data. The input data are in most cases the domain description $T$ therefore, various text processing and filtering techniques, such as RAG, can be implemented here to improve the performance of LLM.