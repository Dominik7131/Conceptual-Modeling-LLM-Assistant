\chapter{LLM-based modeling assistant framework}
\label{chap:framework}

Let us now define a formal LLM-based modeling assistant framework consisting of different types of generators that can be implemented with a different prompting techniques and domain description pre-processing techniques. Furthermore, we define the requirements for these generators and we define their basic meta-structure.


\section{Approach}

We propose solving the automation of the domain modeling steps from the chapter \ref{chap:domain_modeling_process_analysis} as the text generation problems similarly to \citet{Chen2023}, where the authors consider a domain description $T$ with an underlying ground truth model $M$, and approach the problem of model generation as the problem of defining a text generator $f$ with $M' = f(T)$, where $M'$ is the generated model similar to $M$. The generator $f$ is an LLM-based operator defined by a predesigned prompt that instructs an LLM to return a structured textual representation of $M'$ for the input $T$. Based on the generator $f$, we define more fine-grained text generation problems solved by the \emph{text to model generators} that can be used in combination with the \emph{auxiliary generators}.


\section{Definitions of text to model generators}

The following generators solely based on the given domain description suggest some part of the domain model.


\subsection{Domain elements}

The domain element generators are used to generate suggestions for classes, attributes, and associations solely based on the given domain description. Subsequently, these suggestions can be used to \emph{design a class}, \emph{design an attribute for a class}, and \emph{design an association for a class} as described in the domain modeling steps section \ref{sec:modeling_steps}.

\begin{description}
\item [Class generator] $gen_c$ that for $T$ suggests classes $\{C_1, \ldots, C_n\}$ each with a suggested $name(C_i)$.

\item [Attribute generator] $gen_a$ that for $T$ and some class $C$ suggests attributes $\{P_1, \ldots, P_n\}$, each with $source(P_i) = C$, suggested $name(P_i)$, and original text $orig(Pi)$ that is part of $T$ on which base $gen_a$ suggested $P_i$.

\item [Association generator 1] $gen_{r1}$ that for $T$ and a class $C$ suggests associations $\{P_1, \ldots, P_n\}$, each with $source(P_i) = C$ (or $target(Pi) = C$), suggested $name(Pi)$, and with the original text $orig(P_i)$ that is part of $T$ on which base $gen_{r1}$ suggested $P_i$. It also suggests the other class $D$, with $target(Pi)= D$ (or $source(Pi) = D$), and suggested $name(D)$.

\item [Association generator 2] $gen_{r2}$ that for $T$ and a source class $C$ and target class $D$ suggests associations $\{P_1, \ldots, P_n\}$, each with $source(P_i) = C$ and with $target(P_i) = D$, suggested $name(Pi)$, and with the original text $orig(P_i)$ that is part of $T$ on which base $gen_{r2}$ suggested $P_i$.
\end{description}


\subsection{Descriptions}

The description generators aim to suggest description for a given class, attribute, and association solely based on the given domain description. Subsequently, these suggestions can be used to \emph{design a description for a class}, \emph{design a description for an attribute}, and \emph{design a description for an association} as described in the domain modeling steps section \ref{sec:modeling_steps}.

\begin{description}
\item [Class description generator] $gen_{cd}$ that for $T$ and the $name(C)$ of a class $C$ suggests $description(C)$.

\item [Attribute description generator] $gen_{ad}$ that for $T$ and the $name(P)$ of an attribute $P$ suggests $description(P)$.

\item [Association description generator] $gen_{rd}$ that for $T$ and the $name(P)$ of an association $P$ suggests $description(P)$.
\end{description}


\subsection{Data types}

The data type generator is used to suggest the data type of a given attribute solely based on the given domain description. Subsequently, this suggestion can be used to \emph{design a data type for an attribute} as described in the domain modeling steps section \ref{sec:modeling_steps}.

\begin{description}
\item [Attribute data type generator] $gen_{at}$ that for $T$ and the $name(P)$ of an attribute $P$ suggests $dataType(P)$.
\end{description}


\subsection{Cardinalities}

The cardinality generators aim to suggest cardinalities for a given attribute and association solely based on the given domain description. Subsequently, these suggestions can be used to \emph{design a cardinality for an attribute}, and \emph{design a cardinality for an association} as described in the domain modeling steps section \ref{sec:modeling_steps}.

\begin{description}
\item [Attribute cardinality generator] $gen_{ac}$ that for $T$ and the $name(P)$ of an attribute $P$ suggests $cardinality(P)$.

\item [Association cardinality generator] $gen_{rc}$ that for $T$ and the $name(P)$ of an association $P$ suggests $cardinality(P)$.
\end{description}


\section{Definitions of auxiliary generators}

These generators try to support the domain modeling process, for example, by helping to design domain elements and by helping to decide about the next modeling step.


\subsection{Original texts}

When designing a class, an attribute, or an association, the original text generator can be used to suggest a more suitable original text for the corresponding domain element solely based on the given domain description.

\begin{description}
\item [Class original text generator] $gen_{co}$ that for $T$ and the $name(C)$ of a class $C$ suggests $orig(C)$.

\item [Attribute original text generator] $gen_{ao}$ that for $T$ and the $name(P)$ of an attribute $P$ suggests $orig(P)$.

\item [Association original text generator] $gen_{ro}$ that for $T$ and the $name(P)$ of an association $P$ suggests $orig(P)$.
\end{description}


\subsection{Names}

When designing a class, an attribute, or an association, the name generator can be used to suggest a more suitable name for the corresponding domain element solely based on its description and its original text.

\begin{description}
\item [Class name generator] $gen_{cn}$ that for $T$, the $description(C)$ and the $orig(C)$ of a class $C$ suggests  $name(C)$.

\item [Attribute name generator] $gen_{an}$ that for $T$,  a $description(P)$ and an $orig(P)$ of an attribute $P$ suggests $name(P)$.

\item [Association name generator] $gen_{rn}$ that for $T$, the $description(P)$ and the $orig(P)$ of an association $P$ suggests $name(P)$.
\end{description}


\subsection{Summaries}

The summary generators are used to generate a description of the selected part of the domain model. These generators can be used, for example, to better understand the semantics of the given domain model, which can help to decide which next modeling step is needed to model the given domain description.

\begin{description}
\item [Summary plain text generator] $gen_{sp}$ that for a given domain model $M$, and a given style format generates an unstructured domain description $T$ in the given style format.

\item [Summary descriptions generator] $gen_{sd}$ that for a given domain model $M = (\mathcal{C}, \mathcal{P})$ generates a domain description $T$ in form of a list that contains an item for every class $C \in \mathcal{C}$ and for every attribute or association $P \in\mathcal{P}$.
\end{description}


\section{Prompt templates requirements}

A generator can be executed by an LLM call which requires to define a prompt for each generator. For example, the \emph{class generator} can instruct the LLM with a prompt to generate classes. For inserting the user's input into the prompt, it is required to define a prompt template with placeholders that are later on replaced with the user's input. For example, for the \emph{class generator}, the placeholder for the domain description needs to be introduced to insert the user's domain description.

As the LLM output quality greatly depends on the prompt content and the content's order, it is required to create a well-structured prompt and consider suitable prompting techniques. Additionally, the template should instruct the LLM to work within the given context, provide the required task-specific suggestions, and output them in a predefined format for easy parsing.


\section{Meta-template structure}

Based on these generic requirements, we propose the prompt meta-template shown in figure \ref{fig:meta-templates}.

\begin{figure}[!h]
    \centering
\begin{lstlisting}[
  basicstyle=\ttfamily\scriptsize,
  breaklines=true
]
01 Solely based on the given {input data name} {main control instruction}
02 {modeling procedure}
03 {output specification}
04 EXAMPLE START
05 Solely based on the given {input data name} {example main control instruction}
06 This is the given {input data name}: {example context specification}
07 Output: {example output}
08 EXAMPLE END
09 Solely based on the given {input data name} {main control instruction}
10 This is the given {input data name}: {context specification}
11 Output:
\end{lstlisting}
    \caption{\centering Prompt meta-template for the generators}
    \label{fig:meta-templates}
\end{figure}


A generator template is constructed by replacing the placeholders in the meta-template with generator-specific instructions. The meta-template has the following structure.


\subsection{Input data name}

The \emph{input data name} (line 01) provides the name to refer to the data that will be inputted into the prompt, such as the \textit{domain model} for generating summaries based on the given domain model to let the LLM know solely based on which data it should work. When examples are provided, the example input is referenced by the same name, and then it is referenced again before the example input data are provided (line 05 and line 06) and the same is done at the end of the prompt (line 09 and line 10).


\subsection{Main control instruction}

The \emph{main control instruction} (line 01) summarizes the given task, which is, for example, to suggest classes or attributes or associations for a given class $C$ solely based on the given domain description. We place this instruction at the start of the prompt and then in some cases we repeat it at the end of the prompt as the LLM usually puts the most emphasis on the information at the start of the prompt and at the end of the prompt, which we discussed in the section \ref{prompt_general_tips}.


\subsection{Modeling procedure}

The \textit{modeling procedure} (line 02) instructs the LLM on how to proceed before generating the final suggestions. Here, various prompting techniques, such as chain of thoughts, can be implemented.


\subsection{Output specification}

The \textit{output specification} (line 03) defines the required output format so that the output can be automatically parsed.


\subsection{Example specification}

Lines 04-08 represent optional examples. They are all separated from the other instructions with the \texttt{EXAMPLE START} and the \texttt{EXAMPLE END} delimiters. Here, an N-shot prompt design strategy can be implemented, where the prompts contain one or more expected output samples based on concrete sample contexts. Each example starts with the \emph{example main control instruction} (line 05) which is the same as the \emph{main control instruction} but with a concrete class $C$ (and in some cases also with a class $D$) if the task is class-specific. The \emph{example context specification} (line 06) is the concrete context specification for the example and the \emph{example output} (line 07) is the expected output for the given context. The \emph{example output} must correspond to the structure of the \emph{output specification} to make the prompt coherent.


\subsection{Context specification}

The context is specified at the end of the meta-template in the \emph{context specification} (line 10). This context is in most cases the domain description $T$ except for the \emph{summary generators} where the context is some part of the user's domain model. To improve LLM performance, various text processing and filtering techniques, such as RAG, can be implemented here.