\chapter{Related work and possible approaches}


TODO: take inspiration from:
\begin{itemize}
\item related work from our article
\item related work from "Navigating Ontology Development with LLMs"
\item related work from "Automated Domain Modeling with Large Language Models: A Comparative Study"
\item related work from "On the assessment of generative AI in modeling tasks an experience report with ChatGPT and UML" \\~\\
\end{itemize}

Outline:
\begin{itemize}
\item make subsection for each main approach to automatically or semi automatically creating conceptual models (TODO: mainly focus on the approaches that use the LLM)
\begin{itemize}
\item this will probably include the subsection "rule-based" and something like "machine learning"
\end{itemize}
\end{itemize}

\subsection*{Approaches using LLM}
\begin{itemize}
\item generate completed conceptual model: fully automated from domain description as input without user interaction create a conceptual model (Automated Domain Modeling with Large Language Models: A Comparative Study)
\item generate parts of conceptual model: user directly asks some LLM to generate a conceptual model and then the writes following prompts to improve the generated conceptual model (On the assessment of generative AI in modeling tasks an experience report with ChatGPT and UML)
\end{itemize}


\section*{Approaches to creating LLM assistant}

\subsection*{Training own LLM}
- cons: requires a ton of resources: money, time, data


\subsection*{Fine-tuning existing LLM}
- cons: requires a lot of data


\subsection*{Prompt engineering of existing LLM}
- pros: doesn't require that much resources


\section*{Selected approach}

- definition of our problems (entities, attributes and relationships extraction) \\
- (when extracting entities first LLM needs to locate the entity in the text) \\

\section*{How to achieve better results with using an existing LLM}
- RAG (provide LLM only the part of domain description that contains the wanted info) \\
- advanced prompt engineering techniques (CoT, ToT, few-shot prompting) \\



--- Výsledkem této kapitole bude odůvodnění našeho přístupu ---