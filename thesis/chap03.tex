\chapter{Evaluation}

\section{Challenges}
\begin{itemize}
\item one element can be modeled in a various ways
\item for our purposes no tool for automated evaluating
\end{itemize}


\section{Limitations}
\begin{itemize}
\item we used one domain modeling expert for evaluation
\item we did preliminary experiments on most of the configurations with one shorter and one longer domain description and then used more domain descriptions for the most promising configurations
\end{itemize}


\section{Data}
%An~example citation: \cite{Andel07}

In our application we work with domain descriptions to generate suggestions. \\

Data, se kterými pracujeme:
\begin{itemize}
\item popisy domén
\item anotované popisy domén
\end{itemize}


TODO: Možná bude stačit dát k evaluaci sekci o datech, protože toho ve výsledku o těch datech neřeknu moc:
\begin{itemize}
\item jak data vypadají, tedy jednotlivé styly (educational, atd.)
\item možná zmínit náš primární use-case: analytik, který nejprve popis domény příslušným způsobem upraví
\item možná jak byla vytvořena (především ručně, nevygeneroval to LLM)
\item kolik mají tříd, atributů, asociací
\item jeden popis domény používáme pro N-shot prompting
\item zdůvodnit, k čemu máme anotované popisy domén
\end{itemize}



\chapter{Future work}

- dynamic domain description \\

- adding attribute as a relationship (and vice versa) and letting LLM suggest a new appropriate name for this attribute/relationship \\

- translator of user's plain text instructions into changes inside a conceptual model \\

- in domain description option to highlight the part that the assistant should solely focus on \\

- conceptual model updating: assistant that based on some instruction and given conceptual model creates an updated conceptual model (more info: consultation 23) \\

- suggestions generation based on chosen ontology design patterns (more info: consultation 23) \\

