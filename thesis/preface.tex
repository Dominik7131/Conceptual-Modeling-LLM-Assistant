\chapter*{Introduction}
\addcontentsline{toc}{chapter}{Introduction}


\section*{Motivation}

Domain modeling is fundamental both in data and software engineering. Domain modeling techniques, such as conceptual modeling techniques, where a domain model is expressed as an ER or UML class diagram, or ontology engineering techniques, which define a domain model as a domain ontology, are well established both in academia and industry \cite{Verdonck2018}. However, creating these domain models manually takes a non-trivial amount of time and requires trained experts. Therefore, various modeling automation techniques have been proposed \cite{Sonbol2022}. Recently, researchers began to evaluate different approaches to using Large Language Models (LLMs) as possible automation solutions \cite{Chen2023,Saeedizade2024}.


\section*{Solution approach discussion}

Currently, fully automated domain modeling with LLMs works only for simple domain descriptions with a small number of domain elements \cite{Camara2023}. This is true even when advanced prompting techniques are applied \cite{Chen2023,Saeedizade2024}. To mitigate this issue, the user can be introduced into the domain modeling process \cite{Camara2023} and the LLM can be used as an assistant. For example, \citet{Camara2023} experimented with domain modeling in a conversation dialogue fashion with the ChatGPT-3.5. First, the LLM generated some preliminary domain model and then the authors created follow-up instructions to edit their domain models. However, as the LLM executed directly each author's instruction, this frequently led to back-and-forth communication as the authors tried correcting the mistakes made by the LLM in the previous steps. Additionally, there was no option to manually model any parts of the domain model to help the LLM model the parts it struggled with. For a detailed description of related work see chapter \ref{chap:related_work}.


\section*{Using LLM as an assistant}
\label{section:llm_as_an_assistant}

Recently, a lot of applications started using the LLM as an assistant. Here is a list of some of those applications with a brief description of the assistant usage.

\begin{itemize}
\item \textit{Grammarly}\footnote{\url{https://app.grammarly.com/}}: assistant for writing
\item \textit{GitHub Copilot}\footnote{\url{https://github.com/features/copilot}}: assistant for programming
\item \textit{Notion AI}\footnote{\url{https://www.notion.so/help/guides/category/ai}}: assistant for taking notes
\end{itemize}

These applications have the following features in common. First, the LLM usually provides only suggestions to the users. These suggestions usually can be accepted or rejected or in some cases, a new suggestion can be re-generated. The reason for this behaviour is that the LLMs still make a lot of mistakes so direct application of their output is usually not wanted. Typically, these applications explicitly warn their users that the generated output can provide false, biased and outdated information. Furthermore, if these mistakes are not removed by the users these mistakes can accumulate over time and can lead to even worse output performance. Second, the suggestions are usually in a form such that the users can quickly decide if they want to use them. For example, in \textit{Grammarly} the spell-checking feature represents each suggestion as an underlined text that contains a reason why it was underlined and a suggested correction. In \textit{GitHub Copilot} especially in the hands of an experienced developer, it is usually easy to decide if a suggested piece of code or a suggested code documentation is appropriate based on the surrounding context. And finally, in \textit{Notion AI} suggested notes are usually easy to evaluate as they are typically clearly structured.


\section*{Areas with lack of research}

Based on our exploration, the following areas have a lack of research and can have a significant impact on making the domain modeling process faster and more accessible:

\begin{itemize}
\item manual domain modeling extended with the LLM as an assistant that only provides suggestions
\item using retrieval-augmented generation for domain modeling with LLMs
\item using real-life domain descriptions for evaluating domain modeling with LLMs
\item creating a tool for demonstrating the viability of the selected approach
\end{itemize}


\section*{Goals}

The domain modeling process typically follows a systematic approach. First, the domain experts create an unstructured domain description, and second, the modeling experts solely based on this domain description create an exact, structured, and formalized domain model.

This thesis aims to help with the second step and address the mentioned areas with lack of research by creating a tool for domain modeling where the LLM assistant cooperates with a human modeling expert. The modeling expert provides an unstructured domain description as input into the domain modeling process and the assistant generates suggestions solely based on this domain description.

To improve the performance of the LLM, we divide the domain modeling process into a set of simpler steps and we employ the retrieval-augmented generation technique for providing only the relevant parts of the domain description to the LLM based on the given task. This assistant always generates output in the form of suggestions as the LLMs still frequently make mistakes. To help the users decide if they want to apply the generated suggestions, we highlight each suggested domain element in the given domain description. Finally, we evaluate our approach with a real-life domain descriptions.

This LLM assistant is mainly intended to help the users create their domain models faster and to help them better document their domain models. This assistant should be:

\begin{enumerate}
\item intuitive to use
\item suggesting mostly appropriate classes, attributes, and associations
\item suggesting mostly appropriate descriptions of classes, attributes, and associations
\item able to summarize a selected part of a domain model
\item able to highlight in the given domain description the parts that the user already modeled
\item able to generate corresponding suggestions in a few seconds
\end{enumerate}


\section*{Outline}

\noindent{}TODO: Brief description of the thesis chapters