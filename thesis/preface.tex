\chapter*{Introduction}
\addcontentsline{toc}{chapter}{Introduction}


\section*{Motivation}
When a team is working on any project, it is beneficial to unify the language and define the terms 
that everyone will use. This is addressed by creating conceptual models. In a nutshell, conceptual 
models capture classes, their attributes and associations. However, creating these conceptual 
models takes a non-trivial amount of time. \\

Recently, there has been a lot of progress in the development of Large Language Models (LLMs) such 
as ChatGPT. We hypothesize that these LLMs could be used in such a way as to make it easier for 
users to create conceptual models. We know that current LLMs are not smart enough to directly 
create the conceptual model one expects for a given textual description (TODO: cite). However, one possible solution is to use an LLM as an assistant, which we see around us a lot today (e.g. Copilot) (TODO: cite). \\



\section*{Goals}
Aim of this thesis is to create this aforementioned LLM assistant. This assistant is intended to help users to make conceptual models faster and also to help the users to better describe these newly created conceptual models. \\

The main goals are:
\begin{enumerate}
\item experiment with different prompting techniques
\item experiment with different LLMs
\item experiment with different Retrieval augemented generation techniques
\end{enumerate}

TODO: Tady je možná problém, že mluvím o prompting technikách a RAGu předtím než jsem vůbec řekl co to je


\section*{Publication}
We created an article summarizing our main findings and submitted it into the International Conference on Conceptual Modeling ER 2024. (TODO: reference (https://insights.sei.cmu.edu/news/international-conference-on-conceptual-modeling-er-2024-opens-call-for-papers/)?)

\section*{Outline}
