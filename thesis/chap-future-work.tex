\chapter{Future work}

\begin{itemize}
\item removing already existing domain elements which we discussed a little bit in the section \ref{duplicate_domain_elements}
\item summary generation task: experiment with temperature $> 0$
\item problém se snahou vygenerovat pokaždé jiný výstup: nepodařilo se nám přijít na způsob, jak donutit LLM, aby generoval jiný výstup bez toho aniž bychom měnili prompt, future work je asi v tom zkusit nějak nastavit parametry, nebo najít způsob jak to udělat s minimální ztrátou kvality výstupu \\
\item dynamic domain description
\item translator of user's plain text instructions into changes inside a conceptual model (TODO: To možná souvisí s tím dynamickým popisem domény, tedy že budeme umět reagovat na změny)
\item in domain description option to highlight the part that the assistant should solely focus on
\item conceptual model updating: assistant that based on some instruction and given conceptual model creates an updated conceptual model (more info: consultation 23)
\item suggestions generation based on chosen ontology design patterns (more info: consultation 23)
\item dání do promptů buď celého uživatelova konceptuálního modelu, nebo pouze nějakých relevantních částí primárně s tím cílem, aby LLM nenavrhoval sémanticky ekvivalentní návrhy a sekundárně ten LLM třeba může použít ten uživatelův konceptuální model k tomu, aby okoukal styl, kterým uživatel modeluje
\end{itemize}


\section{RAG}
As a future work the syntactic and semantic approach could be combined together. For example, first the semantic approach could filter the domain description and then the syntactic approach could filter the rest or vice versa. Another possibility is to let each approach to assign score to each sentence then combine the scores together and finally, remove the sentences based on some threshold.