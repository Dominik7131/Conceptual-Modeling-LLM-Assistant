\chapter{Conclusion}
\label{chap:conclusion}

In this thesis, we presented an approach to designing conceptual modeling assistants based on Large Language Models (LLMs) that help human modeling experts mainly discover classes, their attributes, and binary associations in the given textual domain description.
Compared with recent pioneering work that utilizes LLMs for domain modeling automation, we proposed a generic framework that mainly contains class, attribute, and association generators that focuses on assisting human modelers during the modeling process instead of suggesting the whole domain model.
We defined the generators as prompt templates that can be configured in different and comparable ways.
On the other hand, we did not focus on more complex ontological constructs such as the most recent \cite{Saeedizade2024}.
We presented concrete baseline configurations of the generic generators and demonstrated their practical applicability.
We also introduced a methodology for detailed experimental evaluation of the performance of the generators.
The precision, recall and F1 scores measured on various real-life domain descriptions written in different styles and with different levels of complexity showed that even our baseline configuration is applicable for practical modeling.
We further supported this conclusion through our user-based evaluation using the prototype tool that showed a high level of acceptance among real users with different specializations and levels of seniority.
Our work can be further extended with more advanced generator configurations and future LLMs.
Our experimental evaluation methodology can be used to measure and compare these advanced configurations.


\section{Future work}


\subsection{Selecting ontology design patterns}

There exist many ontology design patterns\footnote{\url{http://ontologydesignpatterns.org/wiki/Main_Page}} for domain modeling that the users could select before they start with the domain modeling process. However, further research is needed to evaluate the capability of the LLMs to adhere to the selected ontology design patterns. Additionally, it is needed to measure the impact of the use of ontology design patterns on the output quality of the LLMs. 


\subsection{Starting from a smaller domain model}

Currently, the domain modeling process in our application starts from an empty domain model where the user starts by inserting classes, their attributes, and their associations. To make this initial process faster, a smaller domain model could be automatically suggested by the LLM as a starting point.


\subsection{Commands for domain modeling}

To make the domain modeling process faster, the user could input commands in a natural language. These commands could be for example translated by the LLM into the domain modeling steps that we defined in the section \ref{modeling_steps}.


\subsection{Modeling highlighted part of the domain description}

Currently, the whole inserted user's domain description is always processed by our application. To process only some part of the domain description the user has to insert only some part of his domain description. To make this process more user-friendly, only the selected part of the domain description could be processed. This could improve the LLM performance as our application would process only a subset of the original domain description.


\subsection{Removing already modeled suggestions}

The goal is to remove the assistant suggestions of domain elements that the user already has in his domain model. In the section \ref{duplicate_domain_elements} we discussed the main challenge of this problem and we presented our naive approach that removes the domain elements that have the same name as the domain elements in the user's domain model.

Another possible solution is to insert the user's domain model into the prompt and instruct the LLM to not suggest already modeled domain elements. However, this is a challenging task since inserting extra information into the prompt can decrease the LLM's output quality. To improve the performance of the LLM some filtering techniques such as RAG can be used to insert into the prompt only the relevant parts of the user's domain model.


%\section{Improving highlighting of already modeled domain elements}
%
%TODO: Nějak rozumně formulovat, jak by se to dalo udělat, možná toto dám do kapitoly o aplikaci, hlavním future workem je vytvořit zcela nový prompt, aby LLM vykonal tuto úlohu
%
%TODO: How to solve highlighting of not modeled elements: One possible simpler solution is to highlight only the domain elements in the domain description such as we do in the case of generating classes.
%
%For example, consider the following sentence:
%
%\noindent{} ``A person is identified by an ID card or by a driving licence.''
%
%To fully capture the semantics of the attribute \textit{ID card} the LLM would probably generate the original text as the 
%
%of the class \textit{person} it's original text could contain the whole sentence but that means that highlighting this attribute would also highlight the \textit{driving license} which could be an another attribute of the class \textit{person}. However, highlighting only part such as ``\textit{A person is identified by an ID card.}'' does not capture it's semantics \\
%
%\noindent{}TODO: Přeformulovat tak, že začneme nejdřív tím co tu mám napsáno ke konci a potom možným řešením udělat z toho samotný task, kde LLM dostane popis domény a uživatelův konceptuální model a z toho vygeneruje části popisu domény, které nejsou namodelovány v uživatelově konceptuálním modelu plus případně k tomu nějaká úroveň jistoty od 1 do 10, jak moc si LLM myslí, že ten text není namodelován v uživatelově konceptuálním modelu \\
%
%
%\noindent{}TODO: na jaké hlavní problémy jsem narazil a proč některé zvýrazněné prvky v popisu domény ve skutečnosti nemusí být namodelovány a naopak viz poznámky


\subsection{Updating old domain models}

Currently, our application focuses only on creating a domain model from a given domain description. However, the domain description usually changes over time because of new requirements. To help the user quickly update his old domain model a new set of features could be introduced. For example, the old and the new domain description could be inserted to let the assistant suggest changes for updating the old domain model.


\subsection{Summary as a form of documentation}

Currently, the ``summary descriptions'' feature generates a description for each selected domain element as discussed in the section \ref{summarising_domain_model}. This feature could be extended to help the users document their domain model by adding the \textit{accept}, \textit{reject}, and \textit{re-generate} suggestion button to each domain element description so the users could quickly create a description for each selected domain element.


\subsection{RAG models combination}

To improve the performance of our RAG approaches, the syntactic and semantic RAG approaches could be combined. For example, first, the semantic approach could filter the domain description and then the syntactic approach could filter the rest of the text or vice versa. A possible improvement is to let each RAG approach assign some similarity score to each pair: source class and a domain description chunk. Finally, these scores could be combined and used to determine which domain description chunks are relevant for the given source class. \\

\noindent{}TODO: Možnost experimentovat s dalšími jazykovými modely pro sémantický a syntaktický RAG, lze použít coref model pro řešení problému se zájmeny, atd. + možná začít tím, že jsme udělali framework, který lze na několika místech vylepšovat \\

\noindent{}TODO: is-a asociace a možná jestli mě napadne ještě nějaké další dělení (to mi připomnělo, že bych na začátku mohl ukázat, proč není dobré si nechávat vygenerovat atributy a asociace dohromady jako "property", neboli ukážu screenshoty, kde ChatGPT-4o nebude schopen najít triviální asociace) \\

\noindent{}TODO: Divide a expand tlačítka pro návrhy