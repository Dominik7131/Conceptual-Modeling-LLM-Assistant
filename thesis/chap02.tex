\chapter{Methodology}


\section{Prompting methods}

\subsection{What are they}

Něco jako: "In the field of LLMs, a prompt is an input that guides the model’s response
 generation. A prompting technique entails the strategic formulation of these
 prompts to maximize the efficacy of LLMs. It involves the deliberate structuring
 and phrasing of prompts to align with the model’s training and capabilities." (zdroj: Navigating Ontology Development with LLMs)

\subsection{Prompt structure}

We created prompts in form of templates where before sending this prompt to the LLM each symbol is replaced by a given argument.


\subsection{Prompting techniques}
\begin{itemize}
\item subsections: CoT, ToT, few-shot prompting \\
\item CoT can be further divided into more approaches
\end{itemize}



\section{RAG methods}
- TODO: brief explanation of what RAG is, how it works and why it works \\
- RAG approaches: semantic vs. syntactic (more in GAO, Yunfan, et al. Retrieval-augmented generation for large language models: A survey.) \\