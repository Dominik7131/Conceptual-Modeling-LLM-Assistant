%%% Please fill in basic information on your thesis, which will be automatically
%%% inserted at the right places.

% Type of your thesis:
%	"bc" for Bachelor's
%	"mgr" for Master's
%	"phd" for PhD
%	"rig" for rigorosum
\def\ThesisType{mgr}

% Language of your study programme:
%	"cs" for Czech
%	"en" for English
\def\StudyLanguage{cs}

% Thesis title in English (exactly as in the official assignment)
% (Note: \xxx is a "ToDo label" which makes the unfilled visible. Remove it.)
\def\ThesisTitle{Using large language model as an assistant for conceptual modeling}

% Author of the thesis (you)
\def\ThesisAuthor{Dominik Prokop}

% Year when the thesis is submitted
\def\YearSubmitted{2024}

% Name of the department or institute, where the work was officially assigned
% (according to the Organizational Structure of MFF UK in English,
% see https://www.mff.cuni.cz/en/faculty/organizational-structure,
% or a full name of a department outside MFF)
\def\Department{Department of Software Engineering}

% Is it a department (katedra), or an institute (ústav)?
\def\DeptType{Department}

% Thesis supervisor: name, surname and titles
\def\Supervisor{doc. Mgr. Martin Nečaský, Ph.D.}

% Supervisor's department (again according to Organizational structure of MFF)
\def\SupervisorsDepartment{Department of Software Engineering}

% Study programme (does not apply to rigorosum theses)
\def\StudyProgramme{Computer science - Software Systems}

% An optional dedication: you can thank whomever you wish (your supervisor,
% consultant, who provided you with tea and pizza, etc.)
\def\Dedication{I would like to express sincere gratitude to my supervisor doc. Mgr. Martin Nečaský, Ph.D. for his valuable expertise and guidance. And I would like to thank my family for supporting me throughout my whole academic path.
}

% Abstract (recommended length around 80-200 words; this is not a copy of your thesis assignment!)
\def\Abstract{%
This thesis explores the automation of domain modeling using large language models (LLMs), presenting an experimental LLM-based domain modeling assistant that collaborates with a human expert. The assistant provides modeling suggestions based on a given textual description of the domain of interest, aiding in the design of classes, attributes, and associations. We present a generic framework for domain modeling assistants and show how they can be implemented using an LLM. We demonstrate concrete configurations of this framework and their implementations in a prototype application. We evaluated the effectiveness of the framework configurations across various domains. Our findings indicate that the assistant significantly enhances the efficiency of modeling while maintaining a reasonable quality of the outputs.
}

% 3 to 5 keywords (recommended) separated by \sep
% Keywords are useful for indexing and searching for the theses by topic.
\def\ThesisKeywords{%
conceptual modeling\sep large language models\sep AI assistant
}

% If any of your metadata strings contains TeX macros, you need to provide
% a plain-text version for use in XMP metadata embedded in the output PDF file.
% If you are not sure, check the generated thesis.xmpdata file.
\def\ThesisAuthorXMP{\ThesisAuthor}
\def\ThesisTitleXMP{\ThesisTitle}
\def\ThesisKeywordsXMP{\ThesisKeywords}
\def\AbstractXMP{\Abstract}

% If your abstracts are long and do not fit in the infopage, you can make the
% fonts a bit smaller by this setting. (Also, you should try to compress your abstract more.)
\def\InfoPageFont{}
%\def\InfoPageFont{\small}  % uncomment to decrease font size

% If you are studing in a Czech programme, you also need to provide metadata in Czech:
% (in English programmes, this is not used anywhere)

\def\ThesisTitleCS{Použití velkého jazykového modelu jako asistenta pro konceptuální modelování}
\def\DepartmentCS{Katedra softwarového inženýrství}
\def\DeptTypeCS{Katedra}
\def\SupervisorsDepartmentCS{Katedra softwarového inženýrství}
\def\StudyProgrammeCS{Informatika - Softwarové systémy}

\def\ThesisKeywordsCS{%
konceptuální modelování\sep velké jazykové modely\sep AI asistent
}

\def\AbstractCS{%
Tato práce se zabývá automatizací doménového modelování pomocí velkých jazykových modelů (LLMs) a prezentuje experimentálního asistenta využívajícího LLM pro doménové modelování, který spolupracuje s lidským expertem. Tento asistent poskytuje návrhy pro modelování na základě zadaného textového popisu domény a pomáhá při vytváření tříd, atributů a asociací. Zavádíme obecný framework pro asistenty pro doménové modelování a ukazujeme, jak je lze implementovat pomocí LLM. V prototypové aplikaci ukazujeme konkrétní konfigurace tohoto frameworku a jejich implementace, které pak vyhodnocujeme napříč několika doménami. Výsledky naznačují, že náš asistent výrazně zvyšuje efektivitu modelování při zachování rozumné kvality výstupů.
}